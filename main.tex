%% ドキュメントクラスのロード jarticle クラス
%% onecolumn : 一段組み
%% a4j : A4サイズ,和文
%% fleqn : 数式は左揃え
\documentclass[twocolumn,a4j,fleqn]{jarticle}

%% 使用パッケージの指定
%% graphicx : 図の使用
%% fancyhdr : ヘッダ設定
%% lastpage : ページ番号出力マクロ
\usepackage[dvipdfmx]{graphicx}
\usepackage{fancyhdr}
\usepackage{lastpage}
\usepackage{amsmath,amssymb}
\usepackage{bm}
%\usepackage{listings, jlisting}

%\lstset{
%  numbers=left,
%  numbersep=-10pt,
%  columns=l
%}

%% 独自マクロ タイトルスタイルの変更
\makeatletter
\let\@oldmaketitle=\@maketitle
\def\@maketitle{
  \vskip -3zh
  \@oldmaketitle
  %\vspace*{-0.8cm}
}
\makeatother

%% ページのレイアウト指定
\setlength{\textheight}{242mm}
%\setlength{\textwidth}{500pt}
\setlength{\mathindent}{1zw}
\setlength{\headheight}{15.0pt}
\setlength{\voffset}{-1cm}
%\setlength{\hoffset}{-1cm}

%% ヘッダスタイルの適用
\pagestyle{fancy}
\lhead{ 輪講 2014/10/28 }
\chead{}
\rhead{ 高山夏樹 }
\lfoot{}
\cfoot{\thepage}
\rfoot{}
\renewcommand\headrulewidth{0.4pt}
\renewcommand\footrulewidth{0pt}

% タイトル
%\title{
%  {\Large{\bf TeX環境の導入 }}\\
%  \vspace*{4mm}
%  \bf{高山夏樹}\\
%  \vspace*{2mm}
%  {\large 電気通信大学大学院 情報理工学研究科}\\
%}
\title{
  {\Large{\bf 【要約】Connected Segmentation Tree - A Joint Representation of Region Layout and Hierarchy }}
}
\author{\bf{\dag Nerendra Ahuja, \dag Sinisa Todorovic}\\
  %\vspace*{4mm}
  {\large \dag Beckman Institute, University of Illinois at Urbana-Champaign}
}
\date{}

%%----- ここから論文本体 ---------------------------------------
\begin{document}%{
\maketitle

\small{
\begin{abstract}
本論文ではCST(Connected Segmentation Tree)と名付ける新たなオブジェクト表現を
提案する.CSTは光学,幾何学,空間的な隣接関係などの文脈におけるオブジェクトの標準的な特徴を
表現し,オブジェクトを構成する領域特徴を抑制する.
CSTは領域の階層的な関係を表現するST(Segmentation Tree)を拡張することで得られ,
階層関係に加えて階層内の隣接関係を同時に表現することが可能である.
この結果,CSTはRAG(Region Adjacensy Graphs)の階層を表現する.
領域の隣接関係は画素パターンに関するボロノイ図を領域に拡張することで得られる.
あるカテゴリに関するCSTの自動学習はラベル無しのトレーニング画像を用いた
CSTグラフのマッチング問題として形式化され,最大のマッチング部分グラフを合成する
ことで得ることができる.
また,本論文では最も顕著性の高いノードとエッジを同時に探索することで最適化を
行う手法を提案する.ここで,ノードは領域を表し,エッジは領域の隣接関係を表す.
カテゴリモデルの画像のCSTへのマッチングは,カテゴリ内の検出,分割,認識が同時に起きることを
意味する.

\end{abstract}

\section{はじめに}
3次元空間内の物理的な物体はコントラスト,大きさ,形状などの
光学的,および幾何学的な特徴を有しており,かつ,有限であり,凝集している.
これらはまた,階層や抑制を反映した画像構造や,物体を構成する構造の
空間的配置などの特徴を有している.
最後に,物体は空間内で確かな位置を占めている.
自然画像は実際の世界を2次元空間への投影であり,
2次元物体を画像内に持つ.
自然画像内には実世界の構造を模倣した次のような構造が存在する.
\begin{enumerate}
 \item 2次元の領域は空間的な形状や配置を持つ
 \item 3次元の階層構造や,実物体は細かな領域が再帰的に組み込まれているよ
       うに見える.
 \item 2次元の領域は光学的,幾何学的特性を持つ物体によって占有される部分
       画像を構成する.
\end{enumerate}
2次元画像や物体の表現に関する従来の手法の多くは,3番目の構造のみか,
または1と3,2と3番目の組み合わせを用いている.
本論文は画像の構造の観点から上記で述べた3つの構造を同時に表現する
手法を提案する.
また,従来よりも包括的な物体表現手法の利点を,既に存在するカテゴリ分類と
カテゴリ認識の応用を例に示す .

本手法はAhujaらによって提案されているST(Segmentation
Tree)\cite{Ahuja=2007, Todorovic_2006}の拡張に当たる.
STは領域の2番目と3番目の構造をモデル化する.
この手法はマルチスケールで画像を領域分割し,各領域をグラフの
ノードとする.大きなスケールで分割された細かな領域はより小さな
スケールで分割された領域に内包される場合,その領域が示すノードの
子ノードになる.
他の厳密な階層表現と同様にSTも,1番目の構造に関しては細分割領域の中心座標や方向などの一部分を推論することしかできない.これは,2番目と3番目の構造を通して集められた情報から行われる.
しかしながら,STは2番目と3番目の構造に関しては固定されるが,
同じ親ノードに属する細分割領域の集合が空間的にどのよう
に分布するかなどは区別できない.これは図1に示すように外観として異なるも
のが同じとみなされる事を意味する.
本論文で提案する拡張モデルは,この問題について領域間の2次元的な隣接関係
という1番目の構造を考慮することで解決を図る.
その一方で,既にSTで提案されている再帰的な組み込み構造は維持する.
拡張モデルはSTにRAG(Region Adjacency Graphs)を加えたものである.
ここで,RAGは各STノードの子ノードになる.
RAGのエッジは二つの領域が画像空間上で隣り合う場合に張られる.
これはSTにおいては兄弟ノード間にエッジが張られることを意味する.
この操作により,STは二つの異なるエッジ集合を持つグラフへと変換される.
エッジ集合の一つは階層構造の親子関係を示し,もう一方は隣接関係を
表す.このような構造を持つことで,兄弟関係を持たないノードの隣接関係も,
親ノード間の隣接関係を調べることで簡単に検索することができる.
STに隣接関係を加えたものは構造としてはグラフであるが,本論文では
CST(Connected Segmentation Tree)と呼ぶ.
CSTのノードとエッジは属性値を持つことができる.
例えば,ノードやエッジに画像空間上の位置関係に基づいた重みを定義する
ことができる.
このように,CSTはSTをRAGの階層として一般化したものである.
マルチスケールの領域表現が物体カテゴリの基本的な特性として用いられるよう
に,CSTは画像の推論を単純化するための中間的な情報表現として用いることが
できる.

同じ領域集合が異なる空間的な分布を持った場合,2次元物体としては全く異な
るものとなるため,1番目の構造をとらえる隣接関係分布は重要である.
しかしながら,このような分布を定式化することは難しい.
なぜなら領域間において隣接関係が意味するものの明白で直観的な観点がないた
めである.


Ahuja\cite{Ahuja_1982}
Ahuja\cite{Ahuja_1996}
Ahuja\cite{Ahuja_2007}
Bouchard\cite{Bouchard_2005}
Liangliang\cite{Liangliang_2007}
Carneiro\cite{Carneiro_2006}
Hong\cite{Hong_2006}
Crandall\cite{Crandall_2005}
Epshtein\cite{Epshtein_2007}
Fergus\cite{Fergus_2003}
Fidler\cite{Fidler_2007}
Glantz\cite{Glantz_2004}
Ya\cite{Ya_2006}
Levinshtein\cite{Levinshtein_2005}
Ommer\cite{Ommer_2007}
Pelillo\cite{Pelillo_1999}
Russell\cite{Russell_2006}
Shokoufandeh\cite{Shokoufandeh_2006}
Todorovic\cite{Todorovic_2006}
Torsello\cite{Torsello_2003}
Winn\cite{Winn_2005}


\section{従来研究と本手法の利点}


\section{CST}

\subsection{領域の隣接関係}

\subsection{CSTの構築}

% math 1
\begin{equation}
  S_{\widetilde{h}\widetilde{h}'} \triangleq \max_f \Sigma_{(v,v')\in f}(2\min(r_v,r_{v'})-\max(r_v,r_{v'})+1)
\end{equation}


\section{カテゴリの学習}

\subsection{CSTのマッチング}

\subsection{カテゴリからのモデル生成}


\section{評価実験}


\section{おわりに}


%% 参考文献
\bibliographystyle{sieicej}
%\bibliographystyle{jplain}
\bibliography{main}
%\bibliographystyle{sieicej}
%\begin{thebibliography}{99}
%\bibitem{bib:Latex2E}
%  奥村晴彦,改定第4版 LATEX2$\varepsilon$ 美文書作成入門,\\須藤真己(編),pp.1-184,(社)技術評論社,東京,2007.
%\end{thebibliography}
}
\end{document}
